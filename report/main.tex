%%%%%%%%%%%%%%%%%%%%%%%%%%%%%%%%%%%%%%%%%
% University/School Laboratory Report
% LaTeX Template
% Version 4.0 (March 21, 2022)
%
% This template originates from:
% https://www.LaTeXTemplates.com
%
% Authors:
% Vel (vel@latextemplates.com)
% Linux and Unix Users Group at Virginia Tech Wiki
%
% License:
% CC BY-NC-SA 4.0 (https://creativecommons.org/licenses/by-nc-sa/4.0/)
%
%%%%%%%%%%%%%%%%%%%%%%%%%%%%%%%%%%%%%%%%%

%----------------------------------------------------------------------------------------
%	PACKAGES AND DOCUMENT CONFIGURATIONS
%----------------------------------------------------------------------------------------

\documentclass[
    letterpaper, % Paper size, specify a4paper (A4) or letterpaper (US letter)
    10pt, % Default font size, specify 10pt, 11pt or 12pt
]{CSUniSchoolLabReport}

\usepackage[french]{babel}
\usepackage{indentfirst}
\usepackage[pdftex]{hyperref}
\hypersetup{
    colorlinks,
    linkcolor={red!50!black},
    citecolor={blue!50!black},
    urlcolor={blue!80!black}
}
\usepackage{lipsum}
\usepackage{xcolor}
\colorlet{LightGray}{white!95!black}
\usepackage{minted}
\usepackage[nameinlink,noabbrev]{cleveref}
\newcommand{\fig}[1]{\cref{fig:#1}}
\newcommand{\code}[1]{\cref{code:#1}}
\usepackage[justification=centering]{caption}


% Displays a syntax-highlighted code block from a file with caption, that gets
% labelled with its path
% #1 : File with code
% #2 : Language
% #3 : Caption
\newcommand{\Codeblock}[3]{
    \begin{figure}[H]
        \centering
        \inputminted
        [
            linenos,
            frame=lines,
            framesep=2mm,
            breaklines,
            fontsize=\footnotesize,
            bgcolor=LightGray,
            baselinestretch=1
        ]{#2}{#1}

        \captionsetup{singlelinecheck=off}
        \caption{#3}
        \label{code:#1}
    \end{figure}
}

%----------------------------------------------------------------------------------------
%	REPORT INFORMATION
%----------------------------------------------------------------------------------------

\title{Systèmes Temps Réel\\Travaux pratiques} % Report title

\author{Robin \textsc{SHAMSNEJAD}} % Author name(s), add additional authors like: '\& James \textsc{Smith}'

\date{E2I5} % Date of the report

%----------------------------------------------------------------------------------------

\begin{document}

\maketitle

%----------------------------------------------------------------------------------------

\section*{Objectif}

Le but de ce TP est de prendre en main la bibliothèque C++ \textsc{SystemC} afin
de décrire des systèmes temps réel.

\tableofcontents

%----------------------------------------------------------------------------------------

\pagebreak
\section{TP1 Partie A - Analyse d'un exemple}

\subsection{Question A.1}

Après compilation immédiate des fichiers fournis, on obtient le résultat visible
en \code{code/qA1.txt}.

\Codeblock{code/qA1.txt}{bash}{Résultat de la compilation brute}

\subsection{Question A.2}

En lisant le code, on identifie les éléments suivants :

\begin{itemize}
    \item Classe \texttt{fifo} : représente le buffer d'échange entre le
    producteur et le consommateur (channel). Y sont définies les interfaces qui
    permettront d'y entrer et sortir des données.
    \item Classes \texttt{write\_if} et \texttt{read\_if} : interfaces de
    lecture et écriture du channel \texttt{fifo}
    \item Classe \texttt{producer} : Thread d'écriture (process), possède un
    port \texttt{out} capable de se connecter à une interface de type
    \texttt{write\_if}
    \item Classe \texttt{consumer} : Thread de lecture (process), possède un
    port \texttt{in} capable de se connecter à une interface de type
    \texttt{read\_if}.
    \item Classe \texttt{top} : Le module principal
\end{itemize}

\subsection{Question A.3}

Lorsqu'on ajoute un \texttt{wait()} dans le consommateur, on observe qu'il n'a
pas le temps de vider le buffer avant le cycle suivant. Comme on peut le voir
dans la \code{code/qA3a.txt}, il n'a le temps de vider qu'un caractère à chaque
itération.

\Codeblock{code/qA3a.txt}{bash}{Résultat avec un \texttt{wait()} dans le
consommateur}

Lorsque l'on ajoute un autre \texttt{wait()} dans le producteur, cette fois on
observe qu'il n'a pas le temps de remplir le buffer : il n'a le temps d'y écrire
qu'un caractère à chaque fois comme on peut le voir dans la
\code{code/qA3b.txt}.

\Codeblock{code/qA3b.txt}{bash}{Résultat en ajoutant encore un \texttt{wait()},
dans le producteur cette fois-ci}

%----------------------------------------------------------------------------------------

\pagebreak
\section{TP1 Partie B - Modélisation d'un système simple}

\subsection{Question B.1}

En lisant le code, on identifie les éléments suivants :

\begin{itemize}
    \item Classe \texttt{FIFO} : représente le buffer d'échange entre
    \texttt{Bit2Byte} et le consommateur. Y sont définies les interfaces qui
    permettront d'y entrer et sortir des données.
    \item Classes \texttt{write\_if} et \texttt{read\_if} : interfaces de
    lecture et écriture du channel \texttt{FIFO}
    \item Classe \texttt{BitGen} : Thread de génération des bits (process)
    \item Classe \texttt{Bit2Byte} : Thread de conversion des bits en octets
    (process)
    \item Classe \texttt{Consumer} : Thread de consommation des octets générés
    (process)
    \item Classe \texttt{top} : Le module principal
\end{itemize}

\subsection{Question B.2}

On définit les ports dans la classe \texttt{BitGen} comme visible en
\code{code/qB2.txt}. On connecte ensuite ces ports aux canaux dans le
constructeur de \texttt{top} comme visible en \code{code/qB2b.txt}.

\Codeblock{code/qB2a.txt}{cpp}{Module \texttt{BitGen}}
\Codeblock{code/qB2b.txt}{cpp}{Connexion des modules}

\subsection{Question B.3}

L'implémentation de \texttt{thProduce} se fait à l'aide d'une simple lecture de
fichier octet par octet, puis par des décalages de bits successifs, comme on
peut le voir en \code{code/qB3.txt}.

\Codeblock{code/qB3.txt}{cpp}{Thread \texttt{thProduce}}

\subsection{Question B.4}

L'implémentation de la méthode \texttt{read} est relativement triviale et se
trouve en \code{code/qB4.txt}.

\Codeblock{code/qB4.txt}{cpp}{Méthode \texttt{read} de la FIFO}

\subsection{Question B.5}

Enfin, \texttt{thConsume} est une implémentation également relativement simple
et est visible en \code{code/qB5.txt}

\Codeblock{code/qB5.txt}{cpp}{Thread \texttt{thConsume}}

\subsection{Question B.6}

Le résultat final en exécutant le programme donne la sortie visible en
\code{code/qB6.txt}, et on y retrouve bien la phrase "This is a test." du
fichier original.

\Codeblock{code/qB6.txt}{bash}{Sortie du système complet}



%----------------------------------------------------------------------------------------
\pagebreak
\section{TP2 - Modélisation et implémentation d'un système multi-tâches et multi-processeur}

Le but de ce TP est d'étudier la modélisation d'un système exécutant plusieurs
tâches en parallèle, ici dans un système à 3 processeurs exécutant chacun
plusieurs tâches. Dans un premier, temps on s'intéresse ici au système sans
communication entre les processus.

\subsection{Question 1}

En lisant le code on observe les éléments suivants :

\begin{itemize}
    \item Classe \texttt{OS} : représente le système d'exploitation au sein
    duquel s'exécuteront les threads SystemC représentant les processeurs du
    système
    \item Classe \texttt{CPU} : représente un processeur au sein duquel
    s'exécuteront les processus (tâches) pseudo-parallèles
    \item Structure \texttt{Channel} : Canal SystemC pour la communication entre
    les processus
    \item Classe \texttt{Kernel} : Représente le noyau du système d'exploitation
    (lié à la classe \texttt{OS})
\end{itemize}

\subsection{Question 2}

En exécutant le code tel quel, on obtient la sortie visible en FIG

FIG

En modifiant le paramètre consume, on observe FIG 

FIG 

On en conclut donc que le système n'est pas préemptif.

\subsection{Question 3}

\subsubsection{Question 3a}

La création d'une tâche se déroule ainsi :

\begin{enumerate}
    \item La classe \texttt{Top} représentant le système globale est instanciée
    dans \texttt{main()}, et contient 2 instances de \texttt{CPU}
    \item Pour ajouter une tâche à un CPU, on appelle sa méthode
    \texttt{AddInitialTask}
    \item Au sein de la classe CPU se trouve un pointeur vers l'instance de la
    classe \texttt{OS}, dont est appelée la méthode \texttt{RegisterTask}
    \item Cette méthode crée un nouveau contexte utilisateur qui sera manipulé
    par l'instance de \texttt{Kernel}
\end{enumerate}

La destruction se fait ensuite simplement par un `delete` dans le destructeur de
la classe \texttt{OS}.

\subsubsection{Question 3b}

Le corps de la fonction est visible en \code{code/TP2_q3b.txt}, et la tâche est
instanciée dans le \texttt{main()} sur le CPU2 de la même manière que
\texttt{proc4}.

\Codeblock{code/TP2_q3b.txt}{cpp}{Tâche \texttt{proc4}}

\subsubsection{Question 3c}

On instancie une variable statique afin de compter le nombre d'exécutions de
\texttt{proc1}, pour n'instancier \texttt{proc5} qu'à la 5e répétition.

\Codeblock{code/TP2_q3c.txt}{cpp}{Tâche \texttt{proc1} modifiée}

\subsection{Question 4}
\subsubsection{Question 4a}

Le port \texttt{interrupt} permet de simuler une interruption CPU, afin de
pouvoir prioriser les tâches.

\subsubsection{Question 4b-4c}

\textit{Je n'ai malheureusement pas eu le temps de terminer cette partie.}

%----------------------------------------------------------------------------------------
\pagebreak
\section{TP3 - La communication intra-processeur dans un système multi-tâches et multi-processeurs}

\subsection{Question 1}

Cela permet de s'assurer que la communication se fait correctement, quel que
soit le premier processus à bloquer le canal de communication.

\subsection{Question 2}
\subsubsection{Question 2a}

Le corps de la primitive \texttt{ChanIn} est visible en \code{code/TP3_q2a.txt}.

\Codeblock{code/TP3_q2a.txt}{cpp}{Primitive \texttt{ChanIn}}

\subsubsection{Question 2b}

\subsection{Question 3-4}
\textit{Je n'ai malheureusement pas eu le temps de terminer cette partie.}
%----------------------------------------------------------------------------------------

\end{document}